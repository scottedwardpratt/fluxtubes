\documentclass{article}
\usepackage[utf8]{inputenc}
\usepackage{amsmath,amssymb}

\DeclareMathOperator\erf{erf}

\begin{document}

\title{Energy Density Fluctuations in fluxtubes}
\author{Rachel Steinhorst}
\date{\today}
\maketitle

This document is intended to be a summary of what the routine edens.cc in the project fluxtubes does and my thoughts on it.

\section*{Procedure}
The program edens.cc takes as input Amax (the number of gluons), and Ntraj. Given Amax, it calculates all possible random walks of length $\leq$ Amax (CalcPQCount). By this I mean that for each gluon added, a step is taken in the 2-D space representing the possible eigenvalues (p,q) in SU(3), and so the "path" in pq-coordinate space can be seen as a random walk. See the "papers" directory of this project for a description of addition in SU(3). In this case the flux tube in its entirety must form a color-singlet, so when producing trajectories we will only select paths that start and end at (0,0). For each of a set of values of y, it does the following:
\begin{enumerate}
  \item Generate Ntraj*10 trajectories by selecting each step weighted by its degeneracy given that the walk must return to (0,0) (FindTrajectory). Each trajectory is also associated with a random placement of gluons in rapidity space between -7 and 7.
  \item For each random walk, calculate the quadratic Casimir at each step and use this to find the energy density:
  \begin{align*}
  \epsilon = \frac{dE}{dy} &= \frac{1}{\sqrt{2 \pi} \sigma} \int_{-\infty}^{\infty} d\eta \frac{dE}{d\eta} e^{-(\eta-y)^2/2\sigma^2}  \\ 
  &= \frac{1}{2} \sum_a \frac{dE}{d\eta}\Bigr\rvert_{\eta=\eta(a)} \left[ \erf\left(\frac{\eta(a+1)-y}{\sqrt{2}\sigma}\right) - \erf\left(\frac{\eta(a)-y}{\sqrt{2}\sigma}\right)\right].
  \end{align*}
 Here $\sigma$ is the thermal smearing width. We assume that
 \begin{align*}
 \frac{dE}{d\eta} \propto C(a),
 \end{align*}
 therefore in this program we take the constant of proportionality to be 1. All moments are then off by a scalar multiple.
  \item Average over all trajectories (which are split into 10 samples to estimate error) to find $\langle\epsilon^n\rangle$, and therefore the cumulant ratios $\omega, S\sigma, K\sigma^2$.
  \item Write the ratios to an output file (moments.dat), which can be used with moments.py, ratios.py, and altratios.py to graph cumulants and their ratios over y.
\end{enumerate}
The program edens\_vsA.cc is similar, but takes y as an input and lets Amax vary. These results are written to moments\_vsA.dat and can be used with moments\_vsA.py.

\section*{Some Thoughts}
Based on results shown in the figs directory, it appears that the cumulants of energy density fluctuations (or at least for N=1,2) follow the relation
\begin{align*}
C_N = k_N A^N (y^2 - y_0^2)^N
\end{align*}
where $k_N$ are constants and $y_0$ is the maximum/minimum value of rapidity in the flux tube, in this case $\pm 7$. Note that in a Poisson or Gaussian distribution, $C_2 \propto C_1$, whereas here $C_2 \propto C_1^2$. 

The Central Limit Theorem states that the distribution of sufficently large samples of any other distribution will be normal. In heavy ion collisions, this means that for sufficiently high numbers of flux tubes within an event, the measured energy density fluctuations will appear to be normal, despite the distribution itself not being normal. This could be incorporated into the code by incrementing Ebar, E2bar, etc not by edens for every trajectory, but by a sum of edens over every $n$ trajectories. 

Additional fluctuation will also be introduced from fluctuations in flux tube length (by length, here I'm referring to the number of gluons). This could be incorporated by generating the number of gluons for each trajectory according to a Gaussian centered at some average $\overline{A}$. The trick to this is that you'd need to calculate PQCount for the maximum value of $A$, and the larger this value is, the slower the code is going to run while generating all trajectories. Alternately, you could calculate PQCount individually for each trajectory, but this could potentially slow calculation just as much, I'm not sure. Just make sure you call ClearPQCount each time you generate a new one to avoid costly memory leaks.
\end{document}
