\documentclass{article}
\usepackage[utf8]{inputenc}
\usepackage{amsmath,amssymb}

\DeclareMathOperator\erf{erf}

\begin{document}
The program takes as input Amax (the number of gluons), and Ntraj. Given Amax, it calculates all possible random walks of length $\leq$ Amax (CalcPQCount). Then for each of a set of values of y, it does the following:
\begin{enumerate}
  \item Generate Ntraj*10 trajectories by selecting each step weighted by its degeneracy given that the walk must return to (0,0) (FindTrajectory). Each trajectory is also associated with a random placement of gluons in rapidity space between -7 and 7.
  \item For each random walk, calculate the quadratic Casimir at each step and use this to find the energy density:
  \begin{align*}
  \epsilon = \frac{dE}{dy} &= \frac{1}{\sqrt{2 \pi} \sigma} \int_{-\infty}^{\infty} d\eta \frac{dE}{d\eta} e^{-(\eta-y)^2/2\sigma^2}  \\ 
  &= \frac{1}{2} \sum_a \frac{dE}{d\eta}\Bigr\rvert_{\eta=\eta(a)} \left[ \erf\left(\frac{\eta(a+1)-y}{\sqrt{2}\sigma}\right) - \erf\left(\frac{\eta(a)-y}{\sqrt{2}\sigma}\right)\right].
  \end{align*}
 We assume that
 \begin{align*}
 \frac{dE}{d\eta} \propto C(a),
 \end{align*}
 therefore in this program we assume the constant of proportionality is 1. All moments are then off by a scalar multiple.
  \item Average over all trajectories (which are split into 10 samples to estimate error) to find $\langle\epsilon^n\rangle$, and therefore the cumulant ratios $\omega, S\sigma, K\sigma^2$.
  \item Write the ratios to an output file (moments.dat), which can be used with moments.py, ratios.py, and altratios.py to graph cumulants and their ratios over y.
\end{enumerate}
\end{document}
