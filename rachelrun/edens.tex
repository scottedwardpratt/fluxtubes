\documentclass{article}
\usepackage[utf8]{inputenc}
\usepackage{amsmath,amssymb}

\DeclareMathOperator\erf{erf}

\begin{document}
The program takes as input y (the rapidity to calculate energy density fluctuations at), and Ntraj. Then for each of a series of values of the number of gluons Amax, it does the following:
\begin{enumerate}
  \item Calculate all possible random walks of length $\leq$ Amax. (CalcPQCount)
  \item Generate Ntraj*10 trajectories by selecting each step weighted by its degeneracy given that the walk must return to (0,0). (FindTrajectory)
  \item For each random walk, find the quadratic Casimir at each step and use this to find the energy density:
  \begin{align*}
  \epsilon = \frac{dE}{dy} &= \int_{-\infty}^{\infty} d\eta \frac{dE}{d\eta} e^{-(\eta-y)^2/2\sigma^2}  \\ 
  &= \frac{1}{2} \sum_a \frac{dE}{d\eta}\Bigr\rvert_{\eta=\eta(a)} \left[ \erf\left(\frac{\eta(a+1)-y}{\sqrt{2}\sigma}\right) - \erf\left(\frac{\eta(a)-y}{\sqrt{2}\sigma}\right)\right]
  \end{align*}
  \item Average over all trajectories (which are split into 10 samples to estimate error) to find $\langle\epsilon^n\rangle$, and therefore the cumulant ratios $\omega, S\sigma, K\sigma^2$.
  \item Write the ratios to an output file (moments.dat), which can be used with moments.py to graph the ratios over the values of Amax.
\end{enumerate}
\end{document}
